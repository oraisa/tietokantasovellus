
\documentclass[a4paper, 12pt, finnish]{article}

\usepackage{babel}
\usepackage[utf8]{inputenc}
\usepackage[T1]{fontenc}
\usepackage{amsmath}
\usepackage{graphicx}

\title{Muistilistan dokumentaatio}

\author{Ossi Räisä}

\begin{document}

\maketitle

\section{Johdanto}

Muistilistan on tarkoitus auttaa käyttääjää muistamaan kaikki hänelle annetut tehtävät.
Jokainen tehtävä lisätään listaan. Tehtävään kuuluu lyhyt kuvaus, tärkeys, ja mahdollisesti
deadline. Tehtäville voi myös antaa tägin. Tägit periytyvät, eli jokaisella tägillä voi olla
yksi tai useampia vanhempia. Käyttäjä voi hakea omia tehtäviään tägien ja tärkeyden perusteella
ja järjestää tuloksia tärkeyden perusteella. Tehtävän voi poistaa tai merkitä suoritetuksi.
Tehtäviä ja tägejä voi myös muokata jälkeenpäin. Käyttäjien täytyy kirjautua sisään käyttätunnuksella
ja salasanalla.

Muistilista pyörii tietojenkäsittelytieteen laitoksen users-palvelimella ja käyttää php:tä
palvelimen logiikan toteuttamiseen. Käyttöliittymä käyttää javascriptiä, jos se on tarpeen.
Tehtävät ja käyttäjien tiedot tallennetaan PostgreSQL-tietokantaan.

\newpage

\section{Käyttötapaukset}

\begin{figure}[h]
  \caption{Käyttötapauskaavio}
  \centering
  \includegraphics[scale=0.7]{kayttotapauskaavio}
\end{figure}

\subsection{Käyttötapauksien kuvauksia}
\begin{description}
  \item[Tehtävän luonti] \hfill \\
  Tehtävää luodessa käyttäjä voi asettaa tehtävälle tärkeyden, antaa kuvauksen,
  asettaa deadlinen ja antaa tehtävälle tägin.
  \item[Tehtävien listaus] \hfill \\
  Käyttäjä voi listata kaikki tehtävät tai rajata listausta tehtävien tärkeyden,
  deadlinen (kaikki joissa deadline ennen), ja tägien perusteella (kaikki joilla tietty
  tägi tai jokin sen lapsista).
  \item[Tägin luonti] \hfill \\
  Tägin voi luoda uutta tehtävää luodessa, jos uudelle tägille on tarvetta tai
  pääsivulta. Tägiä luodessa käyttäjä antaa nimen sekä mahdolliset vanhemmat.
  Tägiä luodessa on myös mahdollista luoda uusi vanhempi.
  \item[Muita käyttötapauksia] \hfill \\
  Tehtävää tai tägiä muokkaamaan pääsee listaussivulta ja kaikkea, mitä luodessa
  voi asettaa voi muokata. Poisto tapahtuu muokkaussivulta.
\end{description}

\end{document}
